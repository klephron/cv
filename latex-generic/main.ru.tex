%-------------------------
% Resume in Latex
% Author : Matty
% Based on: https://github.com/jakegut/resume (which was based on https://github.com/sb2nov/resume)
% License : MIT
%------------------------

\documentclass[letterpaper,11pt]{article}

\usepackage[T1,T2A]{fontenc}
\usepackage[utf8]{inputenc}

\usepackage{latexsym}
\usepackage[empty]{fullpage}
\usepackage{titlesec}
\usepackage{marvosym}
\usepackage[usenames,dvipsnames]{color}
\usepackage{verbatim}
\usepackage{enumitem}
\usepackage[hidelinks]{hyperref}
\usepackage{fancyhdr}
\usepackage[russian,english]{babel}
\usepackage{tabularx}
\usepackage{xcolor}
\usepackage{fontawesome5}

\input{glyphtounicode}

% -------------------- FONT --------------------
\pagestyle{fancy}
\fancyhf{} % clear all header and footer fields
\fancyfoot{}
\renewcommand{\headrulewidth}{0pt}
\renewcommand{\footrulewidth}{0pt}

% Adjust margins
\addtolength{\oddsidemargin}{-0.5in}
\addtolength{\evensidemargin}{-0.5in}
\addtolength{\textwidth}{1in}
\addtolength{\topmargin}{-1in} % Default was -.5in
\addtolength{\textheight}{1.0in}

\urlstyle{same}

\raggedbottom
\raggedright
\setlength{\tabcolsep}{0in}

% Section formatting
\titleformat{\section}{
  \vspace{-5pt}\scshape\raggedright\large
}{}{0em}{}[\color{black}\titlerule \vspace{-5pt}]

% Subsection formatting
\titleformat{\subsection}{
  \vspace{-4pt}\scshape\raggedright\large
}{\hspace{-.15in}}{0em}{}[\color{black}\vspace{-8pt}]

% Ensure that generate pdf is machine readable/ATS parsable
\pdfgentounicode=1

% -------------------- CUSTOM COMMANDS --------------------
\newcommand{\resumeItem}[1]{
  \item\small{
    {#1 \vspace{-2pt}}
  }
}

\newcommand{\resumeSubheading}[4]{
  \vspace{-2pt}\item
    \begin{tabular*}{0.97\textwidth}[t]{l@{\extracolsep{\fill}}r}
      \textbf{#1} & #2 \\
      \textit{\small#3} & \textit{\small #4} \\
    \end{tabular*}\vspace{-7pt}
}

\newcommand{\resumeSubSubheading}[2]{
    \item
    \begin{tabular*}{0.97\textwidth}{l@{\extracolsep{\fill}}r}
      \textit{\small#1} & \textit{\small #2} \\
    \end{tabular*}\vspace{-7pt}
}

\newcommand{\resumeProjectHeading}[2]{
    \item
    \begin{tabular*}{0.97\textwidth}{l@{\extracolsep{\fill}}r}
      \small#1 & #2 \\
    \end{tabular*}\vspace{-7pt}
}

\newcommand{\resumeExperienceHeading}[2]{
    \item
    \begin{tabular*}{0.97\textwidth}{l@{\extracolsep{\fill}}r}
      \small#1 & #2 \\
    \end{tabular*}\vspace{-7pt}
}

\newcommand{\resumeSubItem}[1]{\resumeItem{#1}\vspace{-4pt}}
\newcommand{\resumeSubHeadingListStart}{\begin{itemize}[leftmargin=0.15in, label={}]}
\newcommand{\resumeSubHeadingListEnd}{\end{itemize}}
\newcommand{\resumeItemListStart}{\begin{itemize}}
\newcommand{\resumeItemListEnd}{\end{itemize}\vspace{-5pt}}

\renewcommand\labelitemii{$\vcenter{\hbox{\tiny$\bullet$}}$}

\setlength{\footskip}{4.08003pt}

% -------------------- START OF DOCUMENT --------------------
\begin{document}

% -------------------- HEADING--------------------
\begin{flushright}
  \vspace{-4pt}
  \color{gray}
  \item
  Last Updated on September 4th, 2023
\end{flushright}

\vspace{-7pt}

\begin{center}
    \textbf{\huge \scshape Никита Кулаков} \\ \vspace{8pt}
    \small 
    \faIcon{phone}
    \href{tel:79243306869}{\underline{+7 924 330 6869}} $  $
    \faIcon{github}
    \href{https://github.com/zubrailx}{\underline{github.com/zubrailx}} $  $
    \faIcon{linkedin}
    \href{https://www.linkedin.com/in/nikita-kulakov-8a21b7203/}{\underline{linkedin.com/nikita-kulakov-8a21b7203}} $  $
    \faIcon{envelope}
    \href{mailto:nv1kulakov@gmail.com}
    {\underline{nv1kulakov@gmail.com}}
\end{center}

% -------------------- EDUCATION --------------------
\section{Образование}
  \resumeSubHeadingListStart
  
    \resumeSubheading
      {Университет ИТМО}{Июнь 2024}
      {Бакалавриат Информатики и Вычислительной Техники}{Средний балл: 4.96/5.00}
      
    \subsection{Соответствующая учебная работа}
      \textbf{Оценивание:} Исключительно на отлично, завершенные курсы\\
      \textbf{Курсы:} Функциональное программирование на языке Haskell (Часть I, II), Математическая логика, Компьютерное зрение, Обработка изображений, Машинное обучение и анализ данных, Прикладная статистика, Встроенные системы, Цифровая схемотехника\\

  \resumeSubHeadingListEnd

% -------------------- SKILLS --------------------
\section{Навыки}
 \begin{itemize}[leftmargin=0.15in, label={}]
    \small{\item{
    
     \textbf{Языки}{: C/C++, Haskell, Python, Lua, Golang, Rust, Verilog, AMD64 Assembly, JavaScript, SQL, \LaTeX} \\
     
     \textbf{Инструменты}{: Git/GitHub, Linux, Bash/Zsh, Makefile, CMake, Conan, Flex, Bison, gRPC, Protobuf, REST, Node.js, PostgreSQL, MongoDB, Docker, Unit, CI/CD, Screen/Tmux, Vim, VS Code, Vivado IDE} \\

     \textbf{Иностранные языки}{: Английский - B2}

    }}
 \end{itemize}
% -------------------- PROJECTS --------------------
\section{Проекты}
    \resumeSubHeadingListStart
    
        \resumeProjectHeading
        {\href{https://github.com/IoT-MHWS}{\underline{\textbf{IoT World Simulation}}} $|$ \footnotesize\emph{C++, CMake, Conan, Python, gRPC, GTest, Protobuf, Docker, Git}}{Май 2023 - Июнь 2023}
        \resumeItemListStart
            \resumeItem{Симулировал реальных мир для устройств сбора данных, реализовав алгоритмы для освещения, температуры, движения воздуха, беспроводной сети с учетом препятствий.}
            \resumeItem{Имплементировал асинхронную многопоточную работу приложения с помощью мьютексов, коллбеков и очередей.}
            \resumeItem{Создал несколько видов виртуальных устройств, умеющих собирать данные с симуляции и отправлять их в озеро данных.}
            \resumeItem{Осуществил обмен между симуляцией и виртуальными устройствами с помощью gRPC посредством одиночных вызовов и потоков.}
        \resumeItemListEnd

        \resumeProjectHeading
        {\href{https://github.com/zubrailx/buffer-lru}{\underline{\textbf{Interactive Buffer LRU}}} $|$ \footnotesize\emph{Verilog, Vivado IDE}}{Апрель 2023}
        \resumeItemListStart
            \resumeItem{Добавил возможность изменения режима работы программы с помощью исключительно двойных хлопков.}
            \resumeItem{Преобразовал данные с PDM интерфейса микрофона и визуализировал амплитуду на LED.}
        \resumeItemListEnd
          
        \resumeProjectHeading
        {\href{https://github.com/zubrailx/academic-relational-db}{\underline{\textbf{Academic Relational DB}}} $|$ \footnotesize\emph{C/C++, Bison, Flex, Boost, gRPC, CTest, Protobuf, Cmake, Bash, Git}}{Январь 2023}
        \resumeItemListStart
            \resumeItem{Реализовал реляционную базу данных на языке C, поддерживающую операции вставки, удаления за $O(1)$, поиска за $O(n)$.}
            \resumeItem{Добавил аллокатор, позволяющий переиспользовать свободные разделы в файле с данными СУБД.}
            \resumeItem{Внедрил выборку с использованием вложенных JOIN с условными вложенными операторами и стандартизовал реализацию новых условных операторов.}
        \resumeItemListEnd

    \resumeSubHeadingListEnd

% -------------------- ACTIVITIES --------------------
\section{Опыт}
  \resumeSubHeadingListStart

        \resumeExperienceHeading
          {\href{https://github.com/zubrailx/tinkoff-hack-2022}{\underline{\textbf{Tinkoff Invest Robot Contest}}} $|$ \footnotesize\emph{Backend Developer, DevOps}}{Май 2022}
        \resumeItemListStart
            \resumeItem{Разработал многопользовательское Java приложение, позволяющее для активных аккаунтов пользователей с заданной периодичностью осуществлять анализ котировок торгуемых бумаг и покупать их по заданному алгоритму.}
            \resumeItem{Настроил Spring Security, Docker контейнеры для развертки фронтенда и бекенда, добавил Nginx.}
        \resumeItemListEnd

        \resumeExperienceHeading
          {\href{https://github.com/zubrailx/ovision-hack-2022}{\underline{\textbf{Ovision Hack Dev Track}}} $|$ \footnotesize\emph{Frontend Developer }}{Апрель 2022}
        \resumeItemListStart
            \resumeItem{Разработал одностраничное JavaScript веб-приложение с использованием React и Effector, считывающее изображение с камеры пользователя, отправляющее его на сервер и визуализирующее точки лица.}
            \resumeItem{Реализовал кадрирование исходного изображения с помощью JS Tensorflow для снижения нагрузки на сеть и сервер.}
        \resumeItemListEnd

    \resumeSubHeadingListEnd 

% -------------------- EXPERIENCE --------------------

\end{document}
