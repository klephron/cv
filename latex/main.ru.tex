%-------------------------
% Resume in Latex
% Author : Matty
% Based on: https://github.com/jakegut/resume (which was based on https://github.com/sb2nov/resume)
% License : MIT
%------------------------

\documentclass[letterpaper,11pt]{article}

\usepackage[T1,T2A]{fontenc}
\usepackage[utf8]{inputenc}

\usepackage{latexsym}
\usepackage[empty]{fullpage}
\usepackage{titlesec}
\usepackage{marvosym}
\usepackage[usenames,dvipsnames]{color}
\usepackage{verbatim}
\usepackage{enumitem}
\usepackage[hidelinks]{hyperref}
\usepackage{fancyhdr}
\usepackage[russian,english]{babel}
\usepackage{tabularx}
\usepackage{xcolor}
\usepackage{fontawesome5}

\input{glyphtounicode}

% -------------------- FONT --------------------
\pagestyle{fancy}
\fancyhf{} % clear all header and footer fields
\fancyfoot{}
\renewcommand{\headrulewidth}{0pt}
\renewcommand{\footrulewidth}{0pt}

% Adjust margins
\addtolength{\oddsidemargin}{-0.5in}
\addtolength{\evensidemargin}{-0.5in}
\addtolength{\textwidth}{1in}
\addtolength{\topmargin}{-1in} % Default was -.5in
\addtolength{\textheight}{1.0in}

\urlstyle{same}

\raggedbottom
\raggedright
\setlength{\tabcolsep}{0in}

% Section formatting
\titleformat{\section}{
  \vspace{-5pt}\scshape\raggedright\large
}{}{0em}{}[\color{black}\titlerule \vspace{-5pt}]

% Subsection formatting
\titleformat{\subsection}{
  \vspace{-4pt}\scshape\raggedright\large
}{\hspace{-.15in}}{0em}{}[\color{black}\vspace{-8pt}]

% Ensure that generate pdf is machine readable/ATS parsable
\pdfgentounicode=1

% -------------------- CUSTOM COMMANDS --------------------
\newcommand{\resumeItem}[1]{
  \item\small{
    {#1 \vspace{-2pt}}
  }
}

\newcommand{\resumeSubheading}[4]{
  \vspace{-2pt}\item
    \begin{tabular*}{0.97\textwidth}[t]{l@{\extracolsep{\fill}}r}
      \textbf{#1} & #2 \\
      \textit{\small#3} & \textit{\small #4} \\
    \end{tabular*}\vspace{-7pt}
}

\newcommand{\resumeSubSubheading}[2]{
    \item
    \begin{tabular*}{0.97\textwidth}{l@{\extracolsep{\fill}}r}
      \textit{\small#1} & \textit{\small #2} \\
    \end{tabular*}\vspace{-7pt}
}

\newcommand{\resumeProjectHeading}[2]{
    \item
    \begin{tabular*}{0.97\textwidth}{l@{\extracolsep{\fill}}r}
      \small#1 & #2 \\
    \end{tabular*}\vspace{-7pt}
}

\newcommand{\resumeExperienceHeading}[2]{
    \item
    \begin{tabular*}{0.97\textwidth}{l@{\extracolsep{\fill}}r}
      \small#1 & #2 \\
    \end{tabular*}\vspace{-7pt}
}

\newcommand{\resumeSubItem}[1]{\resumeItem{#1}\vspace{-4pt}}
\newcommand{\resumeSubHeadingListStart}{\begin{itemize}[leftmargin=0.15in, label={}]}
\newcommand{\resumeSubHeadingListEnd}{\end{itemize}}
\newcommand{\resumeItemListStart}{\begin{itemize}}
\newcommand{\resumeItemListEnd}{\end{itemize}\vspace{-5pt}}

\renewcommand\labelitemii{$\vcenter{\hbox{\tiny$\bullet$}}$}

\setlength{\footskip}{4.08003pt}

% -------------------- START OF DOCUMENT --------------------
\begin{document}

% -------------------- HEADING--------------------
\begin{flushright}
  \vspace{-4pt}
  \color{gray}
  \item
  Last Updated on September 3nd, 2023
\end{flushright}

\vspace{-7pt}

\begin{center}
    \textbf{\huge \scshape Никита Кулаков} \\ \vspace{8pt}
    \small 
    \faIcon{phone}
    \href{tel:79243306869}{\underline{+7 924 330 6869}} $  $
    \faIcon{github}
    \href{https://github.com/zubrailx}{\underline{github.com/zubrailx}} $  $
    \faIcon{linkedin}
    \href{https://www.linkedin.com/in/nikita-kulakov-8a21b7203/}{\underline{linkedin.com/nikita-kulakov-8a21b7203}} $  $
    \faIcon{envelope}
    \href{mailto:nv1kulakov@gmail.com}
    {\underline{nv1kulakov@gmail.com}}
\end{center}

% -------------------- EDUCATION --------------------
\section{Образование}
  \resumeSubHeadingListStart
  
    \resumeSubheading
      {Университет ИТМО}{Июнь 2024}
      {Бакалавриат Информатики и Вычислительной Техники}{Средний балл: 4.96/5.00}
      
    \subsection{Соответствующая учебная работа}
      \textbf{Оценивание:} Исключительно на отлично, завершенные курсы\\
      \textbf{Курсы:} Функциональное программирование на языке Haskell (Часть I, II), Математическая логика, Компьютерное зрение, Обработка изображений, Машинное обучение и анализ данных, Прикладная статистика, Встроенные системы, Цифровая схемотехника\\

  \resumeSubHeadingListEnd

% -------------------- SKILLS --------------------
\section{Навыки}
 \begin{itemize}[leftmargin=0.15in, label={}]
    \small{\item{
    
     \textbf{Языки}{: C/C++, Haskell, Python, Lua, Golang, Rust, Verilog, AMD64 Assembly, JavaScript, SQL, \LaTeX} \\
     
     \textbf{Инструменты}{: Git/GitHub, Linux, Bash/Zsh, Makefile, CMake, Conan, Flex, Bison, gRPC, Protobuf, REST, Node.js, PostgreSQL, MongoDB, Docker, Unit, CI/CD, Screen/Tmux, Vim, VS Code, Vivado IDE}
    }}
 \end{itemize}
% -------------------- PROJECTS --------------------
\section{Проекты}
    \resumeSubHeadingListStart
    
        \resumeProjectHeading
        {\textbf{IoT World Simulation} $|$ \footnotesize\emph{C++, CMake, Conan, Python, gRPC, GTest, Protobuf, Docker, Git}}{May 2023 - June 2023}
        \resumeItemListStart
            \resumeItem{Simulated real world for data collection devices by realizing algorithms for lighting, temperature, air movement, obstacle-aware wireless network.}
            \resumeItem{Implemented asynchronous multi-threaded application operation using mutexes, callbacks, and queues.}
            \resumeItem{Created several kinds of virtual devices that can collect data from the simulation and send it to the data lake.}
            \resumeItem{Performed exchanges between simulation and virtual devices using single calls and streams using gRPC.}
        \resumeItemListEnd

        \resumeProjectHeading
        {\textbf{Interactive Buffer LRU} $|$ \footnotesize\emph{Verilog, Vivado IDE}}{April 2023}
        \resumeItemListStart
            \resumeItem{Added possibility to change the program operation mode using exclusively double claps.}
            \resumeItem{Converted data from PDM microphone interface and visualized amplitude on LED.}
        \resumeItemListEnd
          
        \resumeProjectHeading
        {\textbf{Academic Relational DB} $|$ \footnotesize\emph{C/C++, Bison, Flex, Boost, gRPC, CTest, Protobuf, Cmake, Bash, Git}}{January 2023}
        \resumeItemListStart
            \resumeItem{Implemented a C relational database supporting insert, delete for O(1), search for O(n) operations.}
            \resumeItem{Added an allocator allowing to reuse free partitions in the DBMS data file.}
            \resumeItem{Introduced selection using nested JOINs with conditional nested operators, and standardized the implementation of new conditional operators.}
            \resumeItem{Added processing of syntax and semantic errors with description of reasons.}

        \resumeItemListEnd

        \resumeProjectHeading
        {\textbf{ALU CPU} $|$ \footnotesize\emph{Python, Coverage, Mypy, Pylint, Pytest, CI/CD, Bash, Git}}{December 2022}
        \resumeItemListStart
            \resumeItem{Emulated the operation of a stack processor that fetches instructions and changes the state of register cells, memory and I/O ports.}
            \resumeItem{Implemented the assembler translator into processor instructions in accordance with ISA.}
            \resumeItem{Tested the correctness of the translator and processor emulator using CI/CD on GitHub and GitLab.}
        \resumeItemListEnd
          
    \resumeSubHeadingListEnd

% -------------------- ACTIVITIES --------------------
\section{Опыт}
  \resumeSubHeadingListStart

        \resumeExperienceHeading
          {\textbf{Tinkoff Invest Robot Contest} $|$ \footnotesize\emph{Backend Developer, DevOps}}{May 2022}
        \resumeItemListStart
            \resumeItem{Developed a multi-user Java application that allows active user accounts to analyze quotes of traded stocks and buy them according to a specified algorithm at a given periodicity.}
            \resumeItem{Implemented Spring Security to secure user data API access keys and trading logs on user account.}
            \resumeItem{Configured Docker containers for frontend and backend deployment, added Nginx.}
        \resumeItemListEnd

        \resumeExperienceHeading
          {\textbf{Ovision Hack Dev Track} $|$ \footnotesize\emph{Frontend Developer }}{April 2022}
        \resumeItemListStart
            \resumeItem{Developed a one-page JavaScript web application using React and Effector that reads an image from a user's camera and sends it to a server and visializes facial landmarks on the it.}
            \resumeItem{Implemented cropping of the original image using JS Tensorflow to reduce network and server load.}
        \resumeItemListEnd

    \resumeSubHeadingListEnd 

% -------------------- EXPERIENCE --------------------

\end{document}
