\documentclass[letterpaper,11pt]{article}
\usepackage{preamble}
%-------------------DOCUMENT-------------------%
\hypersetup{
    pdftitle={Кулаков Никита - DevOps},
    pdfauthor={Кулаков Никита},
}
\begin{document}
%-------------------HEADING-------------------%
\begin{table}[h]
  \noindent{%
    \begin{minipage}[t]{0.86\textwidth}
      \vspace{1pt}
      \begin{center}
        \textbf{\Huge\scshape Кулаков Никита Васильевич}
      \end{center}
      \vspace{-6pt}

      Дата рождения: 05.06.2003
      \vspace{2pt}

      Санкт-Петербург, Россия%
      \small
      \hspace{5pt}
      \faIcon{phone} %
      \href{tel:79243306869}{\underline{+7 924 330 6869}}%
      \vspace{6pt}

      \faIcon{telegram} %
      \href{https://t.me/klephron}{\underline{t.me/klephron}}%
      \hspace{10pt}%
      \faIcon{envelope} %
      \href{mailto:klephron@gmail.com}{\underline{klephron@gmail.com}}%
      \hspace{10pt}%
      \faIcon{linkedin} %
      \href{https://www.linkedin.com/in/klephron/}{\underline{linkedin.com/klephron}}%
      \hspace{10pt}%
      \faIcon{github} %
      \href{https://github.com/klephron}{\underline{github.com/klephron}}%
    \end{minipage}%
  }%
  \hfill%
  \noindent{%
    \begin{minipage}[t]{0.13\textwidth}
      \vspace{0pt}
      {%
        \setlength{\fboxsep}{0pt}
        \setlength{\fboxrule}{0.3pt}
        \fbox{\includegraphics[width=2.4cm, height=3.2cm]{face.jpg}}}
    \end{minipage}%
  }%
\end{table}
\vspace{-36pt}

%-------------------ABOUT ME-------------------%
\section{О себе}
Специалист с глубокими знаниями систем Linux, сосредоточенный на их конфигурации и автоматизации процессов. Стремится к детальному анализу технологий для глубокого понимания их внутренних механизмов. Нацелен на развитие в области DevOps через создание эффективной и надежной инфраструктуры.

%-------------------EDUCATION-------------------%
\section{Образование}
\resumeSubHeadingListStart
  \resumeSubheading
    {ИТМО}{Санкт-Петербург, 2026}
    {Магистратура Системное и прикладное программное обеспечение}{5.00/5.00}

  \resumeSubheading
    {ИТМО}{Санкт-Петербург, 2024}
    {Бакалавриат Информатика и вычислительная техника}{4.92/5.00}

\resumeSubHeadingListEnd

%-------------------SKILLS-------------------%
\section{Навыки}
\resumeSubHeadingListStart
  \small{\item{
    \textbf{Технологии}{: Ansible, CMake, Docker, Git, GitHub/GitLab Actions, Grafana, Jenkins, Kafka, KVM, Linux, Make, MongoDB, Nginx, PostgreSQL, RabbitMQ, Redis, Prometheus, Terraform }

    \textbf{Технические навыки}{: Устранение неполадок сети и диагностика, настройка производительности системы}

   \textbf{Языки}{: C/C++, Java (Spring Boot, Spring Cloud, Micronaut), Python, JS/TS, Golang, Bash, Rust, SQL, Protobuf }

   \textbf{Иностранные языки}{: Английский - B2}%
  }}
\resumeSubHeadingListEnd

%-------------------EXPERIENCE-------------------%
\section{Опыт}
\resumeSubHeadingListStart
  \resumeExperienceHeading
    {\textbf{ITMO Events} $|$ \footnotesize\emph{Frontend Team Lead/Developer, DevOps} $|$ практика}{Февраль 2024 - Июнь 2024}

    {\vspace{-10pt}\small Система управления мероприятиями и площадками.\vspace{-7pt}}

  \resumeItemListStart
    \resumeItem{Стандартизировал процесс синхронизации API и разработал адаптер для взаимодействия с приложением, что снизило риски, связанные с обратной совместимостью.}
    \resumeItem{Настроил статический анализ кода и проверки слияния в Github Actions для pull request’ов.}
    \resumeItem{Развернул систему для демонстраций заказчику, устранив при этом множество конфликтов.}
  \resumeItemListEnd

  \resumeExperienceHeading
    {{\underline{\textbf{Bluster Wind}}} $|$ \footnotesize\emph{DevOps, Backend Developer} $|$ стартап, команда из 12 человек}{Сентябрь 2023 - Февраль 2024}

    {\vspace{-10pt}\small Социальная сеть для проведения опросов, создания профилей пользователей на основе их голосований и визуализации результатов по заданным параметрам на графиках.\vspace{-7pt}}

  \resumeItemListStart
    \resumeItem{Настроил сервер разработки с TLS, Nginx, Liquibase, Keycloak, Container Registry и self-hosted DinD runners.}
    \resumeItem{Реализовал пайплайны для тестирования, сборки контейнеров и запуска серверных приложений.}
    \resumeItem{Разработал модуль управления пользователями и подписками для монолитного приложения на Java Spring.}
  \resumeItemListEnd
\resumeSubHeadingListEnd

%-------------------PROJECTS-------------------%
\section{Проекты}
\resumeSubHeadingListStart
  \resumeProjectHeading
    {\href{https://github.com/klephron/natrix}{\underline{\textbf{Natrix Language}}} $|$ \footnotesize\emph{C, Make, Antlr3, Python, Docker}}{Октябрь 2024 - Апрель 2025}

    {\vspace{-10pt}\small Компилируемый, динамический, строго типизированный язык общего назначения. \vspace{-6pt}}

    \resumeItemListStart
      \resumeItem{Увеличил скорость выполнения кода в 3 раза с помощью инструментов профилирования.}
      \resumeItem{Разработал отладчик языка, опираясь на глубокое понимание устройства процессов в Linux.}
    \resumeItemListEnd

  \resumeProjectHeading
  {\href{https://github.com/zubrailx/road-condition-monitoring}{\underline{\textbf{Road Condition Monitoring}}} $|$ \footnotesize\emph{Flutter, Python, Golang, MQTT, Kafka, ClickHouse}}{Февраль 2024 - Июнь 2024}

    {\vspace{-10pt}\small Система отслеживания состояния дорожного покрытия при помощи мобильных устройств водителей.\vspace{-7pt}}

  \resumeItemListStart
    \resumeItem{Обеспечил горизонтальное масштабирование за счет продуманной архитектуры и выбора оптимальных технологий.}
    \resumeItem{Увеличил пропускную способность системы, оптимизировав узкие места, выявленные реализованным нагрузочным тестированием.}
    \resumeItem{Повысил точность предсказаний, внедрив сервисы для тестирования и сбора необработанных данных с датчиков.}
  \resumeItemListEnd

\resumeSubHeadingListEnd
\end{document}
