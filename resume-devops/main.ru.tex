\documentclass[letterpaper,11pt]{article}

\usepackage[T1,T2A]{fontenc}
\usepackage[utf8]{inputenc}

\usepackage{latexsym}
\usepackage[empty]{fullpage}
\usepackage{titlesec}
\usepackage{marvosym}
\usepackage[usenames,dvipsnames]{color}
\usepackage{verbatim}
\usepackage{enumitem}
\usepackage[hidelinks]{hyperref}
\usepackage{fancyhdr}
\usepackage[russian,english]{babel}
\usepackage{tabularx}
\usepackage{xcolor}
\usepackage{fontawesome5}
\usepackage{ragged2e}

\usepackage{graphicx}
\graphicspath{ {./} }

%-------------------PAGE LAYOUT-------------------%
\pagestyle{fancy}
\fancyhf{} % Clear header and footer
\fancyfoot{}
\renewcommand{\headrulewidth}{0pt}
\renewcommand{\footrulewidth}{0pt}

\setlength{\oddsidemargin}{-0.5in}
\setlength{\evensidemargin}{-0.5in}
\addtolength{\textwidth}{1in}
\setlength{\topmargin}{-0.6in}
\addtolength{\textheight}{1.0in}

\setlength{\footskip}{4.08003pt}
%-------------------FORMATTING-------------------%
\urlstyle{same}

\raggedbottom
\raggedright % Left alignment
% \justifying
\setlength{\tabcolsep}{0in}

\titleformat{\section}{
  \vspace{-8pt}\scshape\raggedright\large
}{}{0em}{}[\color{black}\titlerule\vspace{-4pt}]

\titleformat{\subsection}{
  \vspace{-8pt}\scshape\raggedright\large
}{\hspace{-.15in}}{0em}{}[\color{black}\vspace{-8pt}]

% Ensure that generate PDF is machine readable/ATS parsable
\pdfgentounicode=1
%-------------------COMMANDS-------------------%
\urlstyle{same}
\renewcommand\labelitemii{$\vcenter{\hbox{\tiny$\bullet$}}$}

% HEADING
\newcommand{\resumeSubheading}[4]{
  \vspace{-2pt}\item
    \begin{tabular*}{\textwidth}[t]{l@{\extracolsep{\fill}}r}
      \textbf{#1} & #2 \\
      \textit{\small#3} & \textit{\small #4} \\
    \end{tabular*}\vspace{-7pt}
}
\newcommand{\resumeSubHeadingListStart}{\begin{itemize}[leftmargin=0in, label={}]}
\newcommand{\resumeSubHeadingListEnd}{\end{itemize}}

\newcommand{\resumeSubSubheading}[2]{
    \item
    \begin{tabular*}{\textwidth}{l@{\extracolsep{\fill}}r}
      \textit{\small#1} & \textit{\small #2} \\
    \end{tabular*}\vspace{-7pt}
}

\newcommand{\resumeProjectHeading}[2]{
    \item
    \begin{tabular*}{\textwidth}{l@{\extracolsep{\fill}}r}
      \small#1 & #2 \\
    \end{tabular*}\vspace{-7pt}
}

\newcommand{\resumeExperienceHeading}[2]{
    \item
    \begin{tabular*}{\textwidth}{l@{\extracolsep{\fill}}r}
      \small#1 & #2 \\
    \end{tabular*}\vspace{-7pt}
}

% ITEM
\newcommand{\resumeItem}[1]{
  \item\small{
    {#1 \vspace{-2pt}}
  }
}
\newcommand{\resumeItemListStart}{\begin{itemize}}
\newcommand{\resumeItemListEnd}{\end{itemize}\vspace{-5pt}}

\newcommand{\resumeSubItem}[2]{\resumeItem{#1}{#2}\vspace{-4pt}}


%-------------------DOCUMENT-------------------%
\hypersetup{
    pdftitle={Кулаков Никита - DevOps},
    pdfauthor={Кулаков Никита},
}


\begin{document}

%-------------------HEADING-------------------%
\begin{table}[h]
  %\fcolorbox{red}{gray}
  \noindent{%
    \begin{minipage}[t]{0.86\textwidth}
      \vspace{1pt}
      \begin{center}
        \textbf{\Huge\scshape Кулаков Никита Васильевич}
      \end{center}
      \vspace{-6pt}

      Дата рождения: 05.06.2003
      \vspace{2pt}

      Санкт-Петербург, Россия%
      \small
      \hspace{5pt}
      \faIcon{phone} %
      \href{tel:79243306869}{\underline{+7 924 330 6869}}%
      \vspace{6pt}

      \faIcon{telegram} %
      \href{https://t.me/klephron}{\underline{t.me/klephron}}%
      \hspace{10pt}%
      \faIcon{envelope} %
      \href{mailto:klephron@gmail.com}{\underline{klephron@gmail.com}}%
      \hspace{10pt}%
      \faIcon{linkedin} %
      \href{https://www.linkedin.com/in/klephron/}{\underline{linkedin.com/klephron}}%
      \hspace{10pt}%
      \faIcon{github} %
      \href{https://github.com/klephron}{\underline{github.com/klephron}}%
    \end{minipage}%
  }%
  \hfill%
  \noindent{%
    \begin{minipage}[t]{0.13\textwidth}
      \vspace{0pt}
      {%
        \setlength{\fboxsep}{0pt}
        \setlength{\fboxrule}{0.3pt}
        \fbox{\includegraphics[width=2.4cm, height=3.2cm]{face.jpg}}}
    \end{minipage}%
  }%
\end{table}
\vspace{-36pt}

%-------------------ABOUT ME-------------------%
\section{О себе}
Специалист с обширными знаниями в области Linux-систем, с акцентом на конфигурацию и автоматизацию процессов. Ориентируюсь на детальный анализ технологий для глубокого понимания их внутренних механизмов. Стремлюсь к развитию в сфере DevOps, применяя экспертные навыки для создания эффективной и надежной инфраструктуры.

%-------------------EDUCATION-------------------%
\section{Образование}
  \resumeSubHeadingListStart
    \resumeSubheading
      {ИТМО}{Санкт-Петербург, 2026}
      {Магистратура Системное и прикладное программное обеспечение}{5.00/5.00}

    \resumeSubheading
      {ИТМО}{Санкт-Петербург, 2024}
      {Бакалавриат Информатика и вычислительная техника}{4.92/5.00}

  \resumeSubHeadingListEnd

%-------------------SKILLS-------------------%
\section{Навыки}
\resumeSubHeadingListStart
  \small{\item{
    \textbf{Технологии}{: Ansible, CMake, Docker, Git, GitHub/GitLab Actions, Grafana, Jenkins, Kafka, KVM, Linux, Make, MongoDB, Nginx, PostgreSQL, RabbitMQ, Redis, Prometheus, Terraform }

    \textbf{Технические навыки}{: Устранение неполадок сети и диагностика, настройка производительности системы}

   \textbf{Языки}{: C/C++ (Boost), Java (Spring Boot, Spring Cloud, Micronaut), Python, JS/TS, Golang, Bash, Rust, SQL, Protobuf }

   \textbf{Иностранные языки}{: Английский - B2}%
  }}
\resumeSubHeadingListEnd

%-------------------EXPERIENCE-------------------%
\section{Опыт}
  \resumeSubHeadingListStart
    \resumeExperienceHeading
      {\textbf{ITMO Events} $|$ \footnotesize\emph{Team Lead, Frontend Developer, DevOps} $|$ практика}{Февраль 2024 - Май 2024}

      {\vspace{-10pt}\small Система управления мероприятиями и площадками.\vspace{-7pt}}

    \resumeItemListStart
      \resumeItem{Стандартизовал процесс синхронизации API, разработал адаптер для взаимодействия с приложением, что позволило снизить вероятность возникновения рисков, связанных с обратной совместимостью.}
      \resumeItem{Настроил статические анализаторы кода и проверки слияния (merge check) с помощью Github Actions для запросов на внесение изменений (pull request).}
      \resumeItem{Развернул систему для проведения демонстраций заказчику, попутно разрешив множественные конфликты.}
    \resumeItemListEnd

    \resumeExperienceHeading
      {{\underline{\textbf{Bluster Wind}}} $|$ \footnotesize\emph{DevOps, Backend Developer} $|$ стартап, команда из 12 человек}{Октябрь 2023 - Февраль 2024}

      {\vspace{-10pt}\small Социальная сеть для проведения опросов, создания профилей пользователей на основе их голосований и визуализации результатов по заданным параметрам на графиках.\vspace{-7pt}}

    \resumeItemListStart
      \resumeItem{Настроил сервер для разработки, TLS, Nginx, Liquibase, Keycloak, Container Registry, self-hosted DinD runners.}
      \resumeItem{Реализовал пайплайны для тестирования, сохранения контейнеров и запуска серверных приложений, что позволило ускорить разработку на 30 процентов.}
      \resumeItem{Разработал модуль управления пользователями и подписками для монолитного приложения на Java Spring.}
    \resumeItemListEnd

  \resumeSubHeadingListEnd

%-------------------PROJECTS-------------------%
\section{Проекты}
  \resumeSubHeadingListStart
    \resumeProjectHeading
    {\href{https://github.com/zubrailx/road-condition-monitoring}{\underline{\textbf{Road Condition Monitoring}}} $|$ \footnotesize\emph{Flutter, Python, Golang, MQTT, Kafka, ClickHouse}}{Февраль 2024 - Июнь 2024}

      {\vspace{-10pt}\small Система отслеживания состояния дорожного покрытия при помощи мобильных устройств водителей.\vspace{-7pt}}

    \resumeItemListStart
      \resumeItem{Обеспечил горизонтальное масштабирование благодаря продуманной архитектуре и выбору технологий.}
      \resumeItem{Увеличил пропускную способностью системы, оптимизировав узкие места, найденные при помощи проведенного нагрузочного тестирования.}
      \resumeItem{Повысил точность предсказаний за счет использования разработанных сервисов для тестирования и сбора необработанных данных датчиков.}
    \resumeItemListEnd

    \resumeProjectHeading
    {\href{https://github.com/IoT-MHWS}{\underline{\textbf{IoT World Simulation}}} $|$ \footnotesize\emph{C++, CMake, Conan, Python, gRPC, GTest, Protobuf, Docker}}{Май 2023 - Июнь 2023}

      {\vspace{-10pt}\small Система моделирования процессов реального мира для устройств, собирающих состояния окружающей среды и влияющих на саму систему. Взаимодействие с устройствами, выступающими агентами, происходит через gRPC. \vspace{-6pt}}

  \resumeSubHeadingListEnd
\end{document}
