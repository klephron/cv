\documentclass[letterpaper,11pt]{article}
\usepackage{preamble}
%-------------------DOCUMENT-------------------%
\hypersetup{
    pdftitle={Кулаков Никита - DevOps},
    pdfauthor={Кулаков Никита},
}
\begin{document}
%-------------------HEADING-------------------%
\begin{table}[h]
  \noindent{%
    \begin{minipage}[t]{0.86\textwidth}
      {
        \vspace{1pt}
        \begin{center}
          \textbf{\Huge\scshape Кулаков Никита Васильевич}
        \end{center}
        \vspace{-6pt}
      }

      {\small
        Дата рождения: 05.06.2003
        \vspace{2pt}
      }

      {\small
        Санкт-Петербург, Россия
        \hspace{5pt}
        \faIcon{phone}
        \href{tel:79243306869}{\underline{+7 924 330 6869}}
        \vspace{2pt}
      }

      {\small
        \faIcon{telegram}
        \href{https://t.me/klephron}{\underline{t.me/klephron}}%
        \hspace{10pt}%
        \faIcon{envelope}
        \href{mailto:klephron@gmail.com}{\underline{klephron@gmail.com}}%
        \hspace{10pt}%
        \faIcon{linkedin}
        \href{https://www.linkedin.com/in/klephron/}{\underline{linkedin.com/klephron}}%
        \hspace{10pt}%
        \faIcon{github}
        \href{https://github.com/klephron}{\underline{github.com/klephron}}%
      }
    \end{minipage}%
  }%
  \hfill%
  \noindent{%
    \begin{minipage}[t]{0.13\textwidth}
      \vspace{0pt}
      {%
        \setlength{\fboxsep}{0pt}
        \setlength{\fboxrule}{0.3pt}
        \fbox{\includegraphics[width=2.4cm, height=3.2cm]{face.jpg}}}
    \end{minipage}%
  }%
\end{table}
\vspace{-36pt}

%-------------------ABOUT ME-------------------%
\section{О себе}
\resumeHeadingListStart
  \resumeHeadingAbout
  {Специалист с глубокими знаниями GNU/Linux и практическим опытом автоматизации инфраструктуры. Предпочитает глубокое понимание принципов работы технологий для их точной настройки и эффективного применения. Нацелен на развитие в области DevOps через проектирование и поддержку устойчивой и эффективной инфраструктуры.}
\resumeHeadingListEnd

%-------------------EDUCATION-------------------%
\section{Образование}
\resumeHeadingListStart
  \resumeHeadingEducation
    {ИТМО}{Санкт-Петербург, 2026}
    {Магистратура, Системное и прикладное программное обеспечение}{5.00/5.00}
  \resumeHeadingEducation
    {ИТМО}{Санкт-Петербург, 2024}
    {Бакалавриат, Информатика и вычислительная техника - Распределенные системы и сети}{4.92/5.00}
\resumeHeadingListEnd

%-------------------SKILLS-------------------%
\section{Навыки}
\resumeHeadingListStart
  \resumeHeadingSkills
  {Стек}{Linux, Ansible, Terraform, Docker Swarm, Kubernetes, Prometheus, Grafana, ELK, GitHub/GitLab CI/CD, Jenkins, Kafka, RabbitMQ, PostgreSQL, MongoDB, Redis, Nginx, Wireguard, KVM, Make, CMake}

  \resumeHeadingSkills
  {Языки}{Python, Go, Bash, SQL, C/C++, Java (Spring Cloud, Micronaut), JS/TS, Rust, Protobuf}

  \resumeHeadingSkills
  {Иностранные языки}{Английский - B2}
\resumeHeadingListEnd

%-------------------EXPERIENCE-------------------%
\section{Опыт}
\resumeHeadingListStart
  \resumeHeadingExperience
  {Bluster Wind}{Сентябрь 2023 - Февраль 2024}

  \resumeDescription
  {Разработка социальной сети для проведения опросов и визуализации результатов.}

  \resumeItemListStart
    \resumeItem{Контейнеризировал серверные и вспомогательные приложения с использованием Docker, организовал хранение образов в частном Docker Registry.}
    \resumeItem{Реализовал CI/CD-пайплайны в GitHub Actions с self-hosted runners для автоматического тестирования, сборки, релизов и деплоя.}
    \resumeItem{Настроил TLS reverse-proxy на базе Nginx для безопасного доступа к сервисам.}
    \resumeItem{Автоматизировал развертывание инфраструктуры с использованием Ansible и Terraform.}
  \resumeItemListEnd
\resumeHeadingListEnd

%-------------------PROJECTS-------------------%
\section{Проекты}
\resumeHeadingListStart
  \resumeHeadingProject
  {\href{https://github.com/klephron/heatbill}{\underline{HeatBill}}}{Март 2025 - Май 2025}

  \resumeDescription
  {Высоконагруженная система IoT для учета потребления тепла в многоквартирных домах.}

  \resumeItemListStart
    \resumeItem{Разработал Ansible-роли и плейбуки для масштабируемого развертывания приложений в Docker Swarm.}
    \resumeItem{Реализовал централизованный сбор логов в ELK через Filebeat, метрик в Prometheus, дашборды в Grafana.}
    \resumeItem{Настроил шардирование и репликацию MongoDB, Prometheus discovery и Nginx балансировку через Consul.}
  \resumeItemListEnd

  \resumeHeadingProject
  {\href{https://github.com/klephron/me-storage}{\underline{ME Storage}}}{Февраль 2025 - Апрель 2025}

  \resumeDescription
  {Система хранения виртуальных предметов.}

  \resumeItemListStart
    \resumeItem{Разработал модульную Terraform-конфигурацию для создания виртуальных машин на базе KVM.}
    \resumeItem{Настроил Jenkins-пайплайн для сборки, анализа качества кода в SonarQube и деплоя через Ansible.}
    \resumeItem{Создал Ansible-роли и плейбуки для конфигурации Ubuntu-серверов, развертывания multi-node кластера kubeadm, настройки RBAC и деплоя приложений.}
    \resumeItem{Реализовал автоматизированное развертывание SonarQube, Prometheus, Grafana вне кластера и создал RBAC-политику, позволяющую Prometheus собирать метрики сервисов через Kubernetes API.}
  \resumeItemListEnd

  \resumeHeadingProject
  {\href{https://github.com/klephron/road-condition-monitoring}{\underline{Road Condition Monitoring}}}{Февраль 2024 - Июнь 2024}

  \resumeDescription
  {Система отслеживания состояния дорожного покрытия при помощи мобильных устройств водителей.}

  \resumeItemListStart
    \resumeItem{Настроил CI/CD-пайплайны в GitHub Actions для сборки APK и деплоя серверных Go/Python контейнеров.}
    \resumeItem{Реализовал Kafka-centric архитектуру с MQTT Kafka Connector для горизонтального масштабирования.}
    \resumeItem{Повысил пропускную способность системы за счет пакетных вставок в ClickHouse.}
  \resumeItemListEnd
\resumeHeadingListEnd
\end{document}
