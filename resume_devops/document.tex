\hypersetup{
    pdftitle={Кулаков Никита - DevOps},
    pdfauthor={Кулаков Никита},
}

\begin{document}

%-------------------HEADING-------------------%
\begin{table}[h]
  %\fcolorbox{red}{gray}
  \noindent{%
    \begin{minipage}[t]{0.86\textwidth}
      \vspace{1pt}
      \begin{center}
        \textbf{\Huge\scshape Кулаков Никита Васильевич}
      \end{center}
      \vspace{-6pt}

      Дата рождения: 05.06.2003
      \vspace{2pt}

      Санкт-Петербург, Россия%
      \small
      \hspace{5pt}
      \faIcon{phone} %
      \href{tel:79243306869}{\underline{+7 924 330 6869}}%
      \vspace{6pt}

      \faIcon{telegram} %
      \href{https://t.me/zubrailx}{\underline{t.me/zubrailx}}%
      \hspace{10pt}%
      \faIcon{envelope} %
      \href{mailto:nv1kulakov@gmail.com}{\underline{nv1kulakov@gmail.com}}%
      \hspace{10pt}%
      \faIcon{linkedin} %
      \href{https://www.linkedin.com/in/zubrailx/}{\underline{linkedin.com/zubrailx}}%
      \hspace{10pt}%
      \faIcon{github} %
      \href{https://github.com/zubrailx}{\underline{github.com/zubrailx}}%
    \end{minipage}%
  }%
  \hfill%
  \noindent{%
    \begin{minipage}[t]{0.13\textwidth}
      \vspace{0pt}
      {%
        \setlength{\fboxsep}{0pt}
        \setlength{\fboxrule}{0.3pt}
        \fbox{\includegraphics[width=2.4cm, height=3.2cm]{img/face.jpg}}}
    \end{minipage}%
  }%
\end{table}
\vspace{-40pt}
%-------------------EDUCATION-------------------%
\section{Образование}
  \resumeSubHeadingListStart
    \resumeSubheading
      {Университет ИТМО}{Санкт-Петербург, 2026}
      {Магистратура Системное программное обеспечение}{-}
  
    \resumeSubheading
      {Университет ИТМО}{Санкт-Петербург, 2024}
      {Бакалавриат Информатика и вычислительная техника}{4.92/5.00}
      
  \resumeSubHeadingListEnd

%-------------------SKILLS-------------------%
\section{Навыки}
 \begin{itemize}[leftmargin=0.15in, label={}]
    \small{\item{
    
     \textbf{Языки}{: C/C++, Python, Java, JavaScript/TS, Golang, Dart, Bash, Lua, Haskell, Rust, SQL, HTML/CSS, \LaTeX} \\
     
     \textbf{Технологии}{: Ansible, AWS SQS, Bison, CI/CD, ClickHouse, Cloud Platform, CMake, Conan, Docker, Flex, Flutter, Git/GitHub Actions, gRPC, Hazelcast, Jenkins, Kafka, Keycloak, KVM, Linux, Makefile, MongoDB, MQTT, Node.js, Nginx, PostgreSQL, Protobuf, RabbitMQ, Redis, REST, S3, Screen/Tmux, Unit, Vim, Virtual Box, VS Code, YouTrack} \\

     \textbf{Библиотеки}{: Scikit-learn, Tensorflow, MatplotLib, PyTorch, React, Solid, Spring} \\ 
     
     \textbf{Иностранные языки}{: Английский - B2}

    }}
 \end{itemize}


%-------------------EXPERIENCE-------------------%
\section{Опыт}
  \resumeSubHeadingListStart

        \resumeExperienceHeading
          {{\underline{\textbf{Bluster Wind}}} $|$ \footnotesize\emph{Backend Developer, DevOps}}{Октябрь 2023 - Июнь 2024}
        \resumeItemListStart
            \resumeItem{Спроектировал сервис на Golang управления данными о пользователях для формирования рекомендаций и предоставления аналитики опросов в приложении социальной сети.}
            \resumeItem{Разработал модуль управления пользователями и подписками для монолитного приложения на Java Spring.}
            \resumeItem{Настроил сервер для разработки, Keycloak, TLS, сохранение образов, сконфигурировал self-hosted runners и добавил сценарии для тестирования и запуска через GitHub Actions.}
        \resumeItemListEnd

        \resumeExperienceHeading
          {\href{https://github.com/zubrailx/tinkoff-hack-2022}{\underline{\textbf{Tinkoff Invest Robot Contest}}} $|$ \footnotesize\emph{Backend Developer, DevOps}}{Май 2022}
        \resumeItemListStart
            \resumeItem{Разработал многопользовательское Java приложение, позволяющее для активных аккаунтов пользователей с заданной периодичностью осуществлять анализ котировок торгуемых бумаг и покупать их по заданному алгоритму.}
            \resumeItem{Настроил Spring Security, Docker контейнеры для развертки фронтенда и бекенда, Nginx.}
        \resumeItemListEnd

        \resumeExperienceHeading
          {\href{https://github.com/zubrailx/ovision-hack-2022}{\underline{\textbf{Ovision Hack Dev Track}}} $|$ \footnotesize\emph{Frontend Developer }}{Апрель 2022}
        \resumeItemListStart
            \resumeItem{Разработал одностраничное JavaScript веб-приложение с использованием React и Effector, считывающее изображение с камеры пользователя, отправляющее его на сервер и визуализирующее точки лица.}
            \resumeItem{Реализовал кадрирование исходного изображения с помощью JS Tensorflow для снижения нагрузки.}
        \resumeItemListEnd

    \resumeSubHeadingListEnd 


%-------------------PROJECTS-------------------%
\section{Проекты}
  \resumeSubHeadingListStart

    \resumeProjectHeading
    {\href{https://github.com/zubrailx/road-condition-monitoring}{\underline{\textbf{Road Condition Monitoring}}} $|$ \footnotesize\emph{Flutter, Python, Golang, MQTT, Kafka, ClickHouse}}{Февраль 2024 - Июнь 2024}
    \resumeItemListStart
        \resumeItem{Разработал мобильное приложение на Flutter для сбора данных датчиков акселерометра, гироскопа и GPS и визуализации оценок состояния дорожного покрытия на карте.}
        \resumeItem{Обучил и сравнил модели машинного обучения на подготовленном открытом наборе данных датчиков.}
        \resumeItem{Спроектировал сервис предобработки и прогнозирования, буферизированной обработки и API.}
        \resumeItem{Реализовал сервисы для тестирования моделей и сбора необработанных реальных данных.}
    \resumeItemListEnd

    \resumeProjectHeading
    {\href{https://github.com/IoT-MHWS}{\underline{\textbf{IoT World Simulation}}} $|$ \footnotesize\emph{C++, CMake, Conan, Python, gRPC, GTest, Protobuf, Docker}}{Май 2023 - Июнь 2023}
    \resumeItemListStart
        \resumeItem{Симулировал реальных мир для устройств сбора данных, реализовав алгоритмы для освещения, температуры, движения воздуха, беспроводной сети с учетом препятствий.}
        \resumeItem{Имплементировал многопоточную работу приложения с помощью мьютексов, коллбеков и очередей.}
        \resumeItem{Создал виртуальные устройства, умеющие собирать данные с симуляции и отправлять их в озеро данных.}
        \resumeItem{Осуществил обмен между симуляцией и виртуальными устройствами с помощью gRPC посредством одиночных вызовов и потоков.}
    \resumeItemListEnd
      
  \resumeSubHeadingListEnd
\end{document}
